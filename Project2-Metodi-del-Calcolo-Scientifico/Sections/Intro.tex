\section*{Introduzione}


L'obiettivo dell'elaborato è principalmente quello di esporee i risultati ottenuti nella compressione di immagini in scala di grigio attraverso la Cosine Discrete Transform bidimensionale.

La prima parte prevede il confronto tra i risultati, in termini di tempo di esecuzione, tra l'implementazione della DCT-2 di una libreria Open Source, e un'implementazione from scratch.

La seconda parte del progetto consiste nell'implementazione di un'interfaccia grafica che permetta ad un utente di effettuare la compressione di un'immagine secondo parametri personalizzabili.

In quanto all'ambiente di programmazione, è stato scelto Python, in particolare con la librearia dctn.\\
Lo scopo è quello di presentare un'analisi approfondita, facendo assunzioni preliminari sui risultati attesi, per poi analizzare se effettivamente esse corrispondono alla realtà.


\paragraph{Reperibilità del progetto su GitHub}
Nella trattazione che segue verranno descritte le scelte di progettazione effettuate e i corrispettivi codici Python. Invitiamo il lettore a dare uno sguardo alla \href{https://github.com/MarioAvolio/Proj-2-Metodi-Calcolo-Scientifico}{Repository GitHub} qualora si voglia analizzare più dettagliatamente il progetto.